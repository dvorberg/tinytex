\documentclass[a4paper,12pt,english]{article}
\usepackage[main=english,ngerman,latin,greek.polytonic]{babel}

\newcommand{\de}[1]{{\it\foreignlanguage{german}{#1}}\xspace}

\begin{document}

\noindent
\de{Exposé zum Promotionsprojekt}

\vspace{1em}

\doublespacing
\noindent
\Large{\bf Pushing off from Authority – }\\
Pastor Emerson’s Epistemological quest\\
\small{(Working title)}

\vspace{1em}

\noindent
Diedrich Vorberg ‹vorberg@selk.de›

\begin{quote}
  Jonah did the Almighty’s bidding. And what was that, shipmates? \\
  To preach the Truth to the face of Falsehood!
  — H. Melville\footnote{\cite[49]{MobyDick}, (Chapter 9, “The Sermon”)}

  % The faith that stands on authority is not faith. The reliance on
  % authority, measures the decline of religion, the withdrawal of
  % the soul. — The Over-Soul, Collected Works 1:174\footnote{{\it The Collected Works of Ralph Waldo Emerson}, ed. Robert E Spiller et al. 10 vols. to date. Harvard UP, 1971ff. Henceforth abbreviated as “CW vol:page.”}
\end{quote}

\subsection*{Point of Departure}
In the 1830s, authority based on the Bible and ecclesiastical doctrines could no longer serve as the foundation of faith and did not provide certainty in the eyes of Ralph Waldo Emerson. For his teachers the Bible contained historically reliable reports of divine miracles, the basis of their faith. For him the Bible was but historical, lamenting that in conventional preaching “[m]iracles, prophecy, poetry, the ideal life, the holy life, exist as ancient history merely.”\footnote{{\it The Divinity School Address}, vol. 1 p. 80 of {\it The Collected Works of Ralph Waldo Emerson}, ed. Robert E Spiller et al. 10 vols. to date. Harvard UP, 1971ff. Henceforth abbreviated as “CW vol:page.”} For his teachers, the philosophy of John Locke provided plausible proofs of their doctrinal confessions. The Romantic Emerson finds in it nothing but the “corpse-cold Unitarianism of Brattle Street,”\footnote{Cf. note on the {\it Divinity School Address}, 84:35, (note found on page 256 of CW 1).} unable to keep alive the brief sparks that “spring spontaneously from the mind’s own sense of good and fair”\footnote{Cf. \CW[Nature]{1:57}. Quote slightly altered to fit grammatically.} and unsuitable to fuel the fire in the individual’s soul. I believe that these standpoints, as Emerson expressed them in his later philosophical work, are answers to questions that arose during his tenure as prominent preacher in Boston. Privately the deaths of his first wife and his younger brother may have fostered a desire for surety about the immortality of the soul and life ever lasting. Vocationally Emerson considered it his duty as a called and ordained minister to preach to his congregation adequately. He sought beauty through eloquence, good deeds through high morals, and in it all: “the Truth.”

For the later Emerson, author of the famous essays, “the scale to the reason-based hierarchy of consciousness had turned 180 degrees.”\footnote{\de{Die „Bewertung der vernunftorientierten Bewußtseinshierarchie [hatte sich] um 180 Grad gedreht“.} \cite[106]{Peper}.} For him truth did not rest on the tenor of society, history, revelation, or logic, but on the individual’s sensual perceptions and emotional intuitions. Uncultivated nature and human beings unencumbered by culture: those are “symbol of spirit.”\footnote{\CW[Nature]{1:17}} In Emerson’s intellectual development these vehicles gradually became more important than the tenor. They are as significant and, indeed, emancipated epistemologically as the individual is politically in a democratic republic. Hermeneutics, the art of understanding, translation, and exegesis, had lost its primary position to the art of perception, aesthetics.

\end{document}
